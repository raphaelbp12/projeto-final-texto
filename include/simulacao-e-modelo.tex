\chapter{Simulação e Modelo}
\label{ch10}

A simulação do projeto foi feita no ambiente de desenvolvimento de jogos Unity. Ela contém os seguintes componentes: obstáculos fixos e móveis, modelo do unicilo, física e interação com o mundo, circuitos simulados (de teste e de prova), sensores de distância e de velocidade.

\section{Mundo Simulado}
\subsection{Tipos de Circuitos}

\begin{itemize}
  \item apresentacao do circuito de prova
  \item altura das paredes
  \item curvas suaves
  \item curvas acentuadas
  \item features a serem testadas
  \item circuitos de teste
  \item obstaculos mais baixos para encurtar o circuito
\end{itemize}

\subsection{Obstáculos Móveis}

\begin{itemize}
  \item apresentacao
  \item justificativa
  \item nao percebido pelo sensor
  \item trajetoria predeterminada
\end{itemize}

\section{Modelo do Uniciclo}

\begin{itemize}
  \item definicao de uniciclo
  \item justificativa (equacionamento simples, facilidade de prever, existente no lab)
  \item constantes do jogo
  \item friccao
  \item inputs
  \item outputs
\end{itemize}

\section{Sensores}
Os sensores, nessa simulação, são o meio pelo qual o veículo recebe as informações e percebe o ambiente. O desenvolvimento desses componentes foi baseado em sensores de sistemas reais e comerciais, para que a interface entre o sistema e o ambiente simulados seja bem próxima da interface do sistema real. Sendo assim, ficaria possível utilizar o sistema desenvolvido nesse projeto em sistemas reais e obter um resultado semelhante, com poucas alterações.

\subsection{Sensor de Distância}
Para se medir a distância entre o modelo e os obstáculos


\begin{itemize}
  \item sensor real
  \item noise do sensor real
  \item comparacao com sensor simulado
  \item noise do sensor simulado
  \item exibicao do sensor simulado
  \item implementacao
  \item direcoes escolhidas
  \item obstaculos ignorados
\end{itemize}


\subsection{Sensor de Velocidade}


\begin{itemize}
  \item sensor real
  \item noise do sensor real
  \item comparacao com sensor simulado
  \item noise do sensor simulado
  \item exibicao do sensor simulado
  \item implementacao
\end{itemize}