\begin{foreignabstract}
	
	O objetivo desse trabalho é desenvolver um software que faça a interface do Phantom Omni com o ROS (Robot Operating System) que possibilite a teleoperação bilateral de um manipulador com quatro graus de liberdade. A linguagem do software desenvolvido é o C++ e ambas as versões do dispositivo háptico (Phantom Omni e Geomagic Touch) são consideradas. O software também possui uma visualização virtual estimada a partir das medições obtidas do estado do robô e seu modelo URDF. A utilização do software é possível em qualquer sistema operacional que possua o Docker instalado. Os resultados obtidos neste trabalho são validados no TETIS, desenvolvido pelo GSCAR no projeto DORIS, através da teleoperação bilateral com realimentação de força no modo Master-Slave com controle de posição das juntas e velocidade linear do efetuador.
	
\end{foreignabstract}